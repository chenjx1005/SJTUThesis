%# -*- coding: utf-8-unix -*-
%%==================================================
%% thesis.tex
%%==================================================

% 双面打印
\documentclass[master, fontset=adobe, openright, twoside]{sjtuthesis}
% \documentclass[bachelor, fontset=adobe, openany, oneside, submit]{sjtuthesis} 
% \documentclass[master, adobefonts, review]{sjtuthesis} 
% \documentclass[%
%   bachelor|master|doctor,	% 必选项
%   fontset=adobe|windows,  	% 只测试了adobe
%   oneside|twoside,		% 单面打印,双面打印(奇偶页交换页边距,默认)
%   openany|openright, 		% 可以在奇数或者偶数页开新章|只在奇数页开新章(默认)
%   zihao=-4|5,, 		% 正文字号:小四、五号(默认)
%   review,	 		% 盲审论文,隐去作者姓名、学号、导师姓名、致谢、发表论文和参与的项目
%   submit			% 定稿提交的论文,插入签名扫描版的原创性声明、授权声明 
% ]

% 逐个导入参考文献数据库
\addbibresource{bib/thesis.bib}
% \addbibresource{bib/chap2.bib}

\begin{document}

%% 无编号内容:中英文论文封面、授权页
%# -*- coding: utf-8-unix -*-
\title{基于神经网络的大规模在线哈希}
\author{陈均炫}
\advisor{卢宏涛教授}
% \coadvisor{某某教授}
\defenddate{2016年11月20日}
\school{上海交通大学}
\institute{电子信息与电气工程学院}
\studentnumber{1140339015}
\major{计算机科学与技术}

\englishtitle{Large Scale Online Hashing with Neural Network}
\englishauthor{\textsc{Junxuan Chen}}
\englishadvisor{Prof. \textsc{Hongtao Lu}}
% \englishcoadvisor{Prof. \textsc{Uom Uom}}
\englishschool{Shanghai Jiao Tong University}
\englishinstitute{\textsc{Depart of Computer Science and Technology, School of Electronic Information and Electrical Engineering} \\
  \textsc{Shanghai Jiao Tong University} \\
  \textsc{Shanghai, P.R.China}}
\englishmajor{Computer Science and Technology}
\englishdate{Nov. 20th, 2016}


\maketitle

\makeenglishtitle

\makeatletter
\ifsjtu@submit\relax
	\includepdf{pdf/original.pdf}
	\cleardoublepage
	\includepdf{pdf/authorization.pdf}
	\cleardoublepage
\else
	\makeDeclareOriginal
	\makeDeclareAuthorization
\fi
\makeatother


\frontmatter 	% 使用罗马数字对前言编号

%% 摘要
\pagestyle{main}
% !TeX spellcheck = en_US
%# -*- coding: utf-8-unix -*-
%%==================================================
%% abstract.tex for SJTU Master Thesis
%%==================================================

\begin{abstract}

最近几年,互联网上的视觉数据呈现爆炸式的发展,如何快速有效的检索这些数据,是一项迫在眉睫的重要挑战。为了解决这个问题,基于哈希的近似最近邻检索以其在计算速度和存储上的进步,受到了各方面的关注。
其中,局部敏感哈希是一种广为人知的非数据驱动的哈希技术。其使用随机超平面来划分空间构造哈希函数。在局部敏感哈希之后,许多数据驱动的哈希技术被提出.这些方法都是用超平面来划分空间构造哈希函数。
与此同时,其他的一些非线性的哈希也被提出,例如球哈希和K均值哈希。这些哈希技术不基于超平面,而采用非线性的哈希函数,从而相比于线性哈希有更强的划分空间的能力。
尽管上述哈希方法取得了很好的哈希效果,但仍然存在两个重要的缺陷。首先,对于大数据应用,数据存储在分布式硬盘群组中,无法同时读入计算机内存,也就无法采用上述的哈希方法进行处理。第二,互联网中的很多数据以流数据的形式存在,而上述哈希算法需要同时获取全部数据进行处理。
为了解决这些问题,在线哈希技术的概念被提出。在线哈希旨在解决了上述的两个问题,但是,目前存在的在线哈希都是线性哈希技术,相比于非线性哈希方法,效果有所逊色。如何将在线哈希技术与非线性哈希技术相结合成为了热点。
最近,神经网络良好的非线性特征提取能力受到了广泛关注,其有效性得到了很多比赛和真实应用的验证。将神经网络与在线哈希技术相结合,为上述问题打开了新的突破方向。神经网络类型多样,受限玻尔兹曼机,卷积网络,深度信念网络,自组织映射网络等网络模型,各有其在不同任务上的优势。在本篇论文中,我们探索了多种神经网络在哈希任务上的应用方法。同时,提出了在线自组织哈希这一创新的方法。

\keywords{\large 哈希 \quad 在线学习 \quad 神经网络}
\end{abstract}

\begin{englishabstract}

An imperial edict issued in 1896 by Emperor Guangxu, established Nanyang Public School in Shanghai. The normal school, school of foreign studies, middle school and a high school were established. Sheng Xuanhuai, the person responsible for proposing the idea to the emperor, became the first president and is regarded as the founder of the university.

During the 1930s, the university gained a reputation of nurturing top engineers. After the foundation of People's Republic, some faculties were transferred to other universities. A significant amount of its faculty were sent in 1956, by the national government, to Xi'an to help build up Xi'an Jiao Tong University in western China. Afterwards, the school was officially renamed Shanghai Jiao Tong University.

Since the reform and opening up policy in China, SJTU has taken the lead in management reform of institutions for higher education, regaining its vigor and vitality with an unprecedented momentum of growth. SJTU includes five beautiful campuses, Xuhui, Minhang, Luwan Qibao, and Fahua, taking up an area of about 3,225,833 m2. A number of disciplines have been advancing towards the top echelon internationally, and a batch of burgeoning branches of learning have taken an important position domestically.

Today SJTU has 31 schools (departments), 63 undergraduate programs, 250 masters-degree programs, 203 Ph.D. programs, 28 post-doctorate programs, and 11 state key laboratories and national engineering research centers.

SJTU boasts a large number of famous scientists and professors, including 35 academics of the Academy of Sciences and Academy of Engineering, 95 accredited professors and chair professors of the "Cheung Kong Scholars Program" and more than 2,000 professors and associate professors.

Its total enrollment of students amounts to 35,929, of which 1,564 are international students. There are 16,802 undergraduates, and 17,563 masters and Ph.D. candidates. After more than a century of operation, Jiao Tong University has inherited the old tradition of "high starting points, solid foundation, strict requirements and extensive practice." Students from SJTU have won top prizes in various competitions, including ACM International Collegiate Programming Contest, International Mathematical Contest in Modeling and Electronics Design Contests. Famous alumni include Jiang Zemin, Lu Dingyi, Ding Guangen, Wang Daohan, Qian Xuesen, Wu Wenjun, Zou Taofen, Mao Yisheng, Cai Er, Huang Yanpei, Shao Lizi, Wang An and many more. More than 200 of the academics of the Chinese Academy of Sciences and Chinese Academy of Engineering are alumni of Jiao Tong University.

\englishkeywords{\large SJTU, master thesis, XeTeX/LaTeX template}
\end{englishabstract}



%% 目录、插图目录、表格目录
\tableofcontents
\listoffigures
\addcontentsline{toc}{chapter}{\listfigurename} %将插图目录加入全文目录
\listoftables
\addcontentsline{toc}{chapter}{\listtablename}  %将表格目录加入全文目录
\listofalgorithms
\addcontentsline{toc}{chapter}{算法索引}        %将算法目录加入全文目录

\include{tex/symbol} % 主要符号、缩略词对照表

\mainmatter	% 使用阿拉伯数字对正文编号

%% 正文内容
\pagestyle{main}
%# -*- coding: utf-8-unix -*-
%%==================================================
%% chapter01.tex for SJTU Master Thesis
%%==================================================

%\bibliographystyle{sjtu2}%[此处用于每章都生产参考文献]
\chapter{绪论}
\label{chap:intro}

本章节主要介绍哈希技术的研究背景,以及神经网络技术的背景知识。同时,也会介绍全文的结构。

\section{哈希的背景知识}

最近十几年,计算机视觉领域在如何学习能够保持数据相似性的二值编码这个问题上产出了很多工作。\cite{IEEE-13631}\parencite{IEEE-1363}

\section{神经网络的背景知识}

\section{全文结构}



%# -*- coding: utf-8-unix -*-
%%==================================================
%% chapter02.tex for SJTU Master Thesis
%% based on CASthesis
%% modified by wei.jianwen@gmail.com
%% Encoding: UTF-8
%%==================================================

\chapter{ 在线自组织哈希}
\label{chap:SOH}

在哈希领域中,大部分哈希方法如局部敏感哈希(Locality-Sensitive Hashing,LSH)等,使用超平面作为哈希函数。在这些方法中,首先,一些按照特点方式生成的超平面将数据空间切分成几个不重叠的区域;之后,处在每个区域内数据点被量化成一个二值编码。

同时,一些不依靠超平面的哈希方法也被提了出来。如球哈希(Spherical Hashing)和K均值哈希(Kmeans Hashing, KMH)。相比于使用超平面的哈希方法,这些方法的优点在于能定义更加紧密的不重叠区域,从而提升哈希的效果。KMH提出了一个几何视角来比较哈希和量化(Quantization)这两种方法,即任何使用正交的超平面来切割空间的哈希方法,都可以看成是使用一个超正方体的顶点作为量化中心的量化方法。KMH使用EM算法来优化量化误差(quantization error)和亲和误差(affinity error)两种误差的和。凭借K均值量化的适应性和亲和保持能力,KMH的效果超过了绝大部分基于超平面的哈希方法。

虽然这些哈希方法取得了优良的效果,但是他们在两个特点环境下仍存在问题。首先,存储在分布式系统上的海量大数据根本无法存入计算机内存中;其次,很多应用的数据以流式产生,而这些方法必须将所有数据积累起来再学习哈希函数。

为了解决这些问题,两个在线哈希技术被提出,分别是在线哈希(OKH)和在线梗概哈希(OSH)。这两种方法凭借在线学习的方式,解决了上述两个问题。但是,他们是基于超平面的哈希方法,存在着分割空间能力弱的问题。所以,本文在本章节提出一种基于在线自组织映射网络的哈希方法——在线自组织哈希。该哈希方法有两点好处:
 \begin{itemize}
 	\item 该方法假设数据以流式的方法到来,并在收取到任意数据后即可立即更新模型。该方法特别适合数据量或数据维度极大的大数据应用。
 	\item 在构造超正方体的过程中,超正方体顶点的拓扑结构可以被很好的保持。因此,该方法不仅有比基于超平面哈希的优势,还比KMH具有更低的量化误差和亲和误差。
 \end{itemize}

本章首先会介绍与在线自组织哈希相关的工作如K均值哈希、自组织映射网络的相关知识,再详细介绍该方法的算法细节。最后是实验结果和结论部分。

\section{相关工作}
在该部分,本文会简要介绍K均值哈希(Kmeans Hashing,KMH)、自组织映射网络(Self Organizing-Map,SOM)两种技术。

\subsection{K均值哈希}
K均值哈希是一个基于K均值聚类方法的能保持数据与量化中心亲和的哈希方法。该方法结合了哈希与量化技术,非常新颖。
给定一个数据集 $\mathbf{X} = [\mathbf{x}_{1}, \mathbf{x}_{2},...,\mathbf{x}_{n}]^{T} \in R^{n\times d}$ ,
KMH首先构建一系列向量 $\mathbf{W} = [\mathbf{w}_{1},  \mathbf{w}_{2}, ..., \mathbf{w}_{k}]^{T} \in R^{k\times d}$.
该向量集合$\mathbf{W}$ 被叫做码表,其中 $\mathbf{w}_{i}$是一个编码值。每个编码值同时又一个二值码$\mathbf{y}_{i}\in \{0, 1\}^{b}$,其中 $b$ 代表了二值编码的长度,所以可得出$k=2^{b}$。之后KMH将任意数据点$\mathbf{x}$映射到一个编码值$\mathbf{w}_{i(\mathbf{x})}$同时最小化下面的目标函数:
\begin{equation}\label{eq:4}
E = E_{\mathrm{quan}} + \rho E_{\mathrm{aff}}.
\end{equation}
其中$E_{\mathrm{quan}}$是经典的K均值算法的平均\emph{量化误差}:
\begin{equation}\label{eq:1}
E_{\mathrm{quan}} = \frac{1}{n}\sum_{\mathbf{x}\in \mathbf{X}} ||\mathbf{x}-\mathbf{w}_{i(\mathbf{x})}||^{2},
\end{equation}
其中$E_{\mathrm{aff}}$是由于使用二值编码来近似聚类中心所导致的\emph{亲和误差}:
\begin{equation}
E_{\mathrm{aff}}= \sum_{i=1}^{k}\sum_{j=1}^{k}w_{ij}(d(\mathbf{w}_{i}, \mathbf{w}_{j}) - d_{h}(i, j))^2.
\end{equation}
在这里,$w_{ij} = n_{i}n_{j}/n^{2}$. $n_{i}$和$n_{j}$是索引值为$i$和$j$的样本数量。 $d_{h}$是两个二值编码$\mathbf{y}_{i}$和$\mathbf{y}_{j}$的基于海明值的距离:
\begin{equation}\label{eq:d_h}
d_{h}(i,j) \triangleq s \cdot h^{\frac{1}{2}}(\mathbf{y}_{i},\mathbf{y}_{j}),
\end{equation}
其中$s$使用过主成分哈希(PCAH)所确定的一个缩放常量。 $h$代表海明距离。
KMH使用类似于K均值算法中使用的EM算法来优化这个目标函数。

\subsection{自组织映射网络}
\begin{algorithm}[tb]
	\caption{自组织映射算法}
	\label{alg:som}
	\begin{algorithmic}
		\Repeat
		\State 1.在每个时间点$t$, 输入一个$\mathbf{x}(t)$, 并选择距离$\mathbf{x}(t)$最近的神经元。
		\begin{equation}\label{eq:5}
		v(t) = \mathrm{arg\ min}_{1 \leq i \leq k} || \mathbf{x}(t) - \mathbf{w}_{i}(t)||
		\end{equation}
		\State 2.更新所有神经元的权值。
		\begin{equation}\label{eq:6}
		\Delta\mathbf{w}_{i}(t) = \alpha(t)\eta(v, i, t)\big(\mathbf{x}(t) - \mathbf{w}_{i}(t)\big)
		\end{equation}
		\Until{网络收敛}
	\end{algorithmic}
\end{algorithm}
自组织映射网络(SOM)是一个能对高位数据进行拓扑结构保持的有效量化的无监督神经网络。SOM使用一系列的组织在二维平面上的神经元,来组成一个多边形,用来近似$d$维的原始输入数据$\mathbf{X}$的拓扑结构。这里本文假设神经元的维度为$b$ ,共有$k$个神经元。由此在该空间中得到一个向量的集合$\mathbf{R} = [\mathbf{r}_{1},  \mathbf{r}_{2}, ..., \mathbf{r}_{k}]^{T} \in R^{k\times b}$。在学习的初始阶段,所有的权值$\mathbf{W} = [\mathbf{w}_{1},  \mathbf{w}_{2}, ..., \mathbf{w}_{k}]^{T} \in R^{k\times d}$被初始化为随机的数值。其中,$\mathbf{w}_{i}$是神经元$i$的权值向量。该向量和输入数据$\mathbf{x}$的维度$d$一致。然后,自组织映射算法重复算法\ref{alg:som}中的步骤。$\eta(v, i, t)$是邻接函数。一般来说,我们可以使用任意形式的邻接函数,但是在实际问题中,高斯函数是最常用的:
\begin{equation}\label{eq:7}
\eta(v, i, t) = exp\Big(-\frac{Dist(\mathbf{r}_{v} , \mathbf{r}_{i})}{2\sigma(t)^{2}}\Big),
\end{equation}
\begin{equation}\label{eq:8}
\sigma(t) = \sigma_{0}exp(-\frac{t}{\lambda}),
\end{equation}
公式\ref{eq:6}中的学习率$\alpha(t)$同样是一个指数形式的衰减函数,用来确保SOM算法最终收敛:
\begin{equation}\label{eq:9}
\alpha(t) = \alpha_{0}exp(-\frac{t}{\lambda}),
\end{equation}
当SOM的节点数较少时,该算法与K均值算法类似。但SOM的网络很大时,网络的拓扑结构就会呈现出与数据的拓扑结构近似的性质。因此,基于SOM的哈希方法相比于使用一般方法来优化公式\ref{eq:4}的算法,理应具有更优的亲和误差。

\section{算法介绍}
\subsection{在线自组织哈希}
在线自组织哈希(SOH)的研究动机来自SOM算法。SOH以在线自组织的形式来量化特征空间,而不是使用离线的EM算法的形式。并且,SOH并不是采用传统的二维或一维的网络映射,而是将神经元节点构建成一个$b$维的超立方体。 该超立方体具有$k=2^{b}$个顶点,从而每个顶点都有一个唯一的$b$比特位的二值编码$\mathbf{y}_{i}$。该方法在每一步最小化下面的亲和保持采样函数:
\begin{equation}\label{objective_soh}
\begin{split}
E(t) = \sum_{i=1}^{k} \eta(v, i, t) \big( ||\mathbf{x}(t) - \mathbf{w}_{i}||^{2} + \beta  E_{\mathrm{aff}}(\mathbf{w}_{i})\big)
\end{split}
\end{equation}
其中$\beta$是一个固定的权值。
\begin{equation}\label{E_aff_w}
E_{\mathrm{aff}}(\mathbf{w}_{i}) =  \sum_{j=1}^{k}(d(\mathbf{w}_{i}, \mathbf{w}_{j}) - d_{h}(i, j))^{2}
\end{equation}
由此可以推导出算法\ref{alg:som}的第二步中新的更新公式:
\begin{equation}\label{eq_newupdate}
\begin{split}
\Delta\mathbf{w}_{i}(t) = &  \alpha(t)\eta(v, i, t)\big(\mathbf{x}(t) \\ &- \mathbf{w}_{i}(t) -  \beta \frac{\partial E_{\mathrm{aff}}(\mathbf{w}_{i})}{\partial \mathbf{w}_{i}}\big)
\end{split}
\end{equation}
We can derive from Eq.(\ref{E_aff_w}) that:
\begin{equation}\label{eq_partial}
\frac{\partial E_{\mathrm{aff}}}{\partial \mathbf{w}_{i}} = \sum_{j=1}^{k}4 \Big(1 - \frac{d_{h}(i, j)}{d(\mathbf{w}_{i}, \mathbf{w}_{j})} \Big)(\mathbf{w}_{i} - \mathbf{w}_{j})
\end{equation}

In each iteration we adjust the index points according to Eq.(\ref{eq_newupdate}) and Eq.(\ref{eq_partial}). We compute the
distance of two index points in Eq.(\ref{eq:7}) in grid using the square of Hamming distance of their binary indices:
\begin{equation}\label{10}
Dist(\mathbf{r}_{v} , \mathbf{r}_{i}) = h(\mathbf{y}_{v}, \mathbf{y}_{i})^2,
\end{equation}

After learning, we assign each sample $\mathbf{x} \in \mathbf{X}$ a binary code as the binary indices of the nearest index point.
\include{tex/example}
\include{tex/faq}
\include{tex/summary}

\appendix	% 使用英文字母对附录编号,重新定义附录中的公式、图图表编号样式
\renewcommand\theequation{\Alph{chapter}--\arabic{equation}}	
\renewcommand\thefigure{\Alph{chapter}--\arabic{figure}}
\renewcommand\thetable{\Alph{chapter}--\arabic{table}}
\renewcommand\thealgorithm{\Alph{chapter}--\arabic{algorithm}}

%% 附录内容,本科学位论文可以用翻译的文献替代。
\include{tex/app_setup}
\include{tex/app_eq}
\include{tex/app_cjk}
\include{tex/app_log}

\backmatter	% 文后无编号部分 

%% 参考资料
\printbibliography[heading=bibintoc]

%% 致谢、发表论文、申请专利、参与项目、简历
%% 用于盲审的论文需隐去致谢、发表论文、申请专利、参与的项目
\makeatletter

%%
% "研究生学位论文送盲审印刷格式的统一要求"
% http://www.gs.sjtu.edu.cn/inform/3/2015/20151120_123928_738.htm

% 盲审删去删去致谢页
\ifsjtu@review\relax\else
  \include{tex/ack} 	  %% 致谢
\fi

\ifsjtu@bachelor
  % 学士学位论文要求在最后有一个英文大摘要,单独编页码
  \pagestyle{biglast}
  \include{tex/end_english_abstract}
\else
  % 盲审论文中,发表学术论文及参与科研情况等仅以第几作者注明即可,不要出现作者或他人姓名
  \ifsjtu@review\relax
    \include{tex/pubreview}
    \include{tex/projectsreview}  
  \else
    \include{tex/pub}	      %% 发表论文
    \include{tex/projects}  %% 参与的项目
  \fi
\fi

% \include{tex/patents}	  %% 申请专利
% \include{tex/resume}	  %% 个人简历

\makeatother

\end{document}
